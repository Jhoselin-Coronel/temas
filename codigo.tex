\documentclass{article}
\usepackage[spanish]{babel}
\usepackage{amsmath,amsthm,amsfonts}
\usepackage{latexsym,amssymb}
\usepackage{xcolor}
\usepackage[utf8]{inputenc}
\usepackage[dvips]{graphicx}
\usepackage{float}
\usepackage{natbib}
\newtheorem{teor}{Teorema}
\title{Numeros naturales y numeros figurados }
\author{Jhoselin Coronel}
\date{29 de abril de 2024}
\begin{document}
\maketitle
\section{Numeros naturales y subconjuntos de interes}
En lo que sigue vamos a encontrar con frecuencia cada una de las clases de numeros que hemos revisado. Sin embargo, vamos a enfocar nuestra atencion principalmente en los numeros naturales ( algo menos en los enteros ), y en propiedades y subconjuntos de interes de $\mathbb{N}_0$.\cite{vazquez2017numeros}
\begin{teor}[Algoritmo de la division]\label{uno}
Sean $a$ y $b$ números naturales cualesquiera $\left ( b> 0 \right )$.Existen dos numeros únicos $q$ y $r$, llamados cociente y residuo, respectivamente, tales que:
\begin{align*}
a= bq+r \quad con & \quad 0\leq r< b& 
\end{align*}
\end{teor}

\begin{proof}[Demostracion]
Dado $a$ y $b$ números naturales con $b>0$, podemos escribir 
$a$ en términos de $b$ como $a=bq+r$, donde 
$q$ es el cociente y 
$r$ es el residuo. Queremos demostrar que existe un par único $q$ y $r$ que satisfaga esta ecuación,con $0\leq r< b$. Primero, veamos que existe al menos un par $q$ y $r$ que satisface la ecuación. Podemos demostrarlo utilizando el principio de la división entera.
Despues, probamos la unidad del par $q$ y $r$ .Supongamos que hay dos pares $q_1$ y $r_1$, $q_2$ y $r_2$ que satisfagan la ecuacion. Restando estas ecuaciones, obtenemos $b(q_1-q_2)=r_1-r_2$. Dado que $0\leq r_1< b$ y $0\leq r_2< b$, la única posibilidad es que $r_1= r_2$, lo que implica  $q_1= q_2$. Por lo tanto, hemos demostrado que existe un unico par $q$ y $r$ que satisfacen la ecuacion $a=bq+r$ con $0\leq r< b$
\end{proof}
\subsection{particiones}
Cuando somos ninos, uno de los primeros hechos que aprendemos respecto a los numeros naturales es que este conjunto se puede separar en pares e impares:
$$\mathbb{N}_0$$
\begin{center}
\begin{tabular}{|c|c|c|}
\hline
\color{red} \bf pares & \color{red} \bf impares \\
\hline  
\bf 0& \bf 1 \\
\hline
\bf 2 & \bf 3\\
\hline
\bf 4 & \bf 5\\
\hline
\bf 6 & \bf 7\\
\hline
\bf 8 & \bf 9\\
\hline
\bf ... & \bf ...\\
\hline
\bf 2q & \bf 2q+1\\
\hline
\end{tabular}
\end{center}
\section{Numeros figurados}
En matemáticas, un número figurado es todo número natural que, al ser representado por un conjunto de puntos equidistantes, puede formar una figura geométrica regular.\cite{rosas2008numeros}

\subsection{Numeros triangulares}
\begin{description}
\item[Formula:] $t_n= \frac{n\left ( n+1 \right )}{2}$
\end{description}
\begin{figure}[H]
\centering
\includegraphics[scale=0.8]{t1.jpg}
\end{figure}
\subsection{Numeros cuadrados}
\begin{description}
\item[Formula:] $c_n= n^{2}$
\end{description}
\begin{figure}[H]
\centering
\includegraphics[scale=0.8]{c1.jpg}
\end{figure}
\subsection{Numeros pentagonales}
\begin{description}
\item[Formula:] $p_n=\frac{n\left ( 3n-1 \right )}{2}$
\end{description}
\begin{figure}[H]
\centering
\includegraphics[scale=0.5]{pt.jpg}
\end{figure}
\bibliographystyle{plain}
\bibliography{citasapa}
\end{document}
